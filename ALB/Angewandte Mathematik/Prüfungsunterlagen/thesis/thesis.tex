\documentclass[12pt,twoside]{report}   % oder {book} ohne twoside
\usepackage[german,ngerman]{babel}     % korrekte Trennung allgemein
\usepackage{ucs}                       % korrekte Behandlung von äöüÄÖÜß
\usepackage[utf8x]{inputenc}           % korrekte Behandlung von äöüÄÖÜß
\usepackage[T1]{fontenc}               % korrekte Trennung bei ä,ö,ü,...
\usepackage{helvet,courier,mathptmx}   % korekte Fonts für pdf
\usepackage{textpos,xcolor,rawfonts,latexsym,graphicx,epsf,verbatim} 
\usepackage[dvipdfm,bookmarksopen,backref,hyperindex,colorlinks,linkcolor=mycolorc,pdfauthor={Prof. Dr. Burkhard Lenze},pdfsubject={Vorschlag Bachelor-/Master-/Arbeit-Lay-Out},pdftitle={Implementierung und Anwendung neuronaler Netze unter Java},pdfkeywords={Neuronen, neuronale Netze, Java}]{hyperref} % entweder dvips oder dvipdfm

\pagestyle{myheadings}
\textheight23cm
\textwidth14cm
\voffset0cm
\topskip0cm
\topmargin-1.2cm
\headheight1.0cm
\headsep1.5cm
\oddsidemargin1.0cm
\evensidemargin1.0cm
\renewcommand{\baselinestretch}{1.4} 

% ***************************** EIGENE MACROS ******************************** 
\newcommand{\mytoday}{13.\ März 2007} 
\newcommand {\cpr}{Ferdi Student, \today}
\newtheorem{theo}{Satz}[section]
\newtheorem{defi}{Definition}[section]
\newtheorem{prog}{Programm}[section]
\def\eop{{ \vrule height7pt width7pt depth0pt}\par\bigskip}
\def\meop{\hfill {\vrule height7pt width7pt depth0pt}}
\newcommand{\RR}{\mathbf{R}}
\newcommand{\NN}{\mathbf{N}}
\newcommand{\QQ}{\mathbf{Q}}
\newcommand{\ZZ}{\mathbf{Z}}
\newcommand{\CC}{\mathbf{C}}
\def\a{{\vec a}}
\def\b{{\vec b}}
\def\c{{\vec c}}
\def\d{{\vec d}}
\def\e{{\vec e}}
\def\h{{\vec h}}
\def\i{{\vec i}}
\def\x{{\vec x}}
\def\t{{\vec t}}
\def\w{{\vec w}}
\def\0{{\vec 0}}
\def\J{{\vec J}}
\def\j{{\vec j}}

\definecolor{mycolorc}{cmyk}{0,0.5,1,0}
\newfont{\smc}{cmss12 scaled 2000}
\newcommand{\fhdohbig}%
  {{\begin{color}{mycolorc} \smc\noindent Fachhochschule \end{color}
      \hfill \smc Fachbereich \\[0.4cm] 
    \begin{color}{mycolorc} \smc Dortmund \end{color} \hfill \smc Informatik}}

\definecolor{fh_orange}{rgb}{0.953,0.201,0}
\definecolor{fh_grau}{rgb}{0.76,0.75,0.76}
\sloppy

\begin{document}

\begin{titlepage}
%%%%%%%%%%%%%%%%%%%%%%%%%%%%%% -*- Mode: Latex -*- %%%%%%%%%%%%%%%%%%%%%%%%%%%%
%% 
%% pa_ba_titelblatt.tex 
%% 
%% Copyright (C) 2008 Alexander Sprack / Claudia Holz
%% 
%%%%%%%%%%%%%%%%%%%%%%%%%%%%%%%%%%%%%%%%%%%%%%%%%%%%%%%%%%%%%%%%%%%%%%%%%%%%%%%
  \begin{textblock}{6.5}(-1.5,-3)
    \begin{color}{fh_grau}
      \rule{6.8cm}{33cm}    
    \end{color}
  \end{textblock}
  \begin{textblock}{6.5}(-1.2,-0.7)
%  \includegraphics[width=3.8cm]{my-fh-logo}% selbst basteln, falls gewünscht! 
                                            % Das offizielle Logo ist nicht
                                            % gestattet!! Bitte BEACHTEN!!!
  \end{textblock}
  \begin{textblock}{6.5}(-1.3,1)
    {\large \textsf{Bachelor-/Masterarbeit}}            
  \end{textblock}

  \begin{textblock}{7}(4.1,2)
    {\noindent \huge 
      \textsf{\textbf{hier der Titel der Arbeit\\[0.3cm] 
          \Large  Zweizeiliger Untertitel\\[0.05cm]
          sofern vorhanden}} }
  \end{textblock}


  \begin{textblock}{6}(4.1,6.5)\noindent
    \textsf{An der Fachhochschule Dortmund\\
    im Fachbereich Informatik\\
    Studiengang Informatik \\
    erstellte Bachelor-/Masterarbeit \\
    zur Erlangung des akademischen Grades\\
    Bachelor/Master of Science}
  \end{textblock}

  \begin{textblock}{6.5}(-0.9,10.8)
    \noindent
    \textsf{von \\
      Ferdi Student \\
      geb.\ am xx.xx.xxxx  \\
      Matr.-Nr. xxxx xxx\\[0.7cm]
      Betreuer: Prof.\ Dr.\ Burkhard Lenze \\[0.5cm]
      Dortmund, \today}    
  \end{textblock}

\end{titlepage}

\newpage

\setcounter{page}{1}
\pagenumbering{roman}
{\baselineskip=15pt \tableofcontents}
{\baselineskip=22pt \listoffigures}

\newpage

\chapter*{Überblick}
\section*{Kurzfassung}
Hier sollte eine halbseitige Kurzfassung der Arbeit stehen.
\vfill\vfill\vfill\vfill\vfill\vfill
\section*{Abstract}
Here, an abstract written in English should appear.
\vfill\vfill\vfill\vfill\vfill\vfill
\pagebreak

\chapter{Einleitung}
\label{chap1}
\markboth{\ \ \ \  {\tiny \cpr} \hfill Kap.\ \ref{chap1} \ \ Einleitung}{Kap.\ \ref{chap1} \ \ Einleitung \hfill {\tiny \cpr} \ \ \ \ }
\pagenumbering{arabic}

\section{Motivation}
{\bf WICHTIGER HINWEIS:} 
Alles in diesem {\bf Entwurf} den Aspekt
{\bf wissenschaftliches Arbeiten} bei der Anfertigung einer
Bachelor-/Master-Arbeit 
Betreffende ist absolut {\bf verbindlich} und muss uneingeschränkt
berücksichtigt werden! \\[0.1cm]
Alles in diesem {\bf Entwurf} das 
{\bf Layout} einer Bachelor-/Master-Arbeit Betreffende gibt die 
{\bf persönliche} Meinung des 
Verfassers wieder! Jede Studentin und jeder Student mag eigene Vorstellungen
entwickeln! Hier findet man eventuell erste hilfreiche Anhaltspunkte. \\[0.1cm]
Nun zum Überblick! Was sollte er u.\ a.\ enthalten:
\begin{itemize}
\itemsep -6pt
\item Erläuterung der Problemstellung 
\item Motivation für die Beschäftigung mit dem Problem
\item Hinweis auf eventuell schon vorhandene Entwicklungen 
\item Abgrenzung der eigenen Arbeit von eventuell schon vorhandenen
      Entwicklungen 
\item kurzer Abriss über den Inhalt der Arbeit
\end{itemize}

\section{Notation}
In diesem Abschnitt wird die in dieser Arbeit verwendete Notation 
vorgestellt und erläutert, falls sie relativ aufwendig ist.

\section{Grundlagen}
Wohlbekannte, aber für diese Arbeit besonders wichtige Resultate finden
sich hier, wenn sie denn benötigt werden. Wichtig in diesem und in allen
folgenden Kapiteln: {\bf Zitate angeben!!!} Wann immer etwas aus einem Buch,
einer Veröffentlichung, einem Vortrag oder einer www-Page entnommen ist, 
{\bf muss} dies durch eine entsprechende Angabe der Quelle im Text (z.B.\ vgl.\
\cite{len2}, S.\ 23ff.) sowie einer Angabe der vollständigen Quelle im
Literaturverzeichnis kenntlich gemacht werden! Es ist absolut unzulässig,
längere Passagen {\bf wörtlich} oder {\bf sinngemäß und nahezu wörtlich} aus
einem anderen Dokument zu übernehmen, ohne dies präzise zu zitieren,
auch wenn man die Referenz pauschal im Literaturverzeichnis angibt
(Plagiat, nicht bestanden, keine Wiederholung möglich). Im Rahmen der 
Erklärung am Ende der Bachelor-/Master-Arbeit verpflichtet sich die
Studentin bzw.\ der 
Student, dieser, im Rahmen wissenschaftlicher Arbeit fundamentalen Pflicht,
alle Quellen angegeben zu haben und Zitate kenntlich gemacht zu haben,
nachgekommen zu sein. \\
Überwiegt bei einer Arbeit der Anteil korrekt
zitierter, aber mehr oder weniger wörtlich übernommener Passagen, ist sie zwar
im engeren Sinne kein Plagiat, allerdings auch kein Nachweis einer gemäß
Prüfungsordnung zu 
erbringenden selbstständigen wissenschaftlichen und fachpraktischen Leistung
(nicht bestanden, Wiederholung möglich). 

\chapter{Theorie neuronaler Netze}
\label{chap2}
\markboth{\ \ \ \  {\tiny \cpr} \hfill Kap.\ \ref{chap2} \ \ Theorie neuronaler Netze}{Kap.\ \ref{chap2} \ \ Theorie neuronaler Netze \hfill {\tiny \cpr} \ \ \ \ }
\section{Die formalen Neuronen}
In diesen Abschnitten geht es um die mathematischen Grundlagen der zu
behandelnden Problemstellung. Hier können Definitionen und Sätze
auftauchen, wobei die Sätze teils bewiesen werden sollten (falls einfach
und für das weitere Verständnis wesentlich) oder ihre Beweise sauber
zitiert werden. Beispiele: \\
\begin{defi}
{\rm (Formales Neuron)} \\
Ein formales Neuron ist eine Funktion
$\kappa: \RR^n \to \RR^m$, die erzeugt wird durch die Verkettung 
einer sogenannten {\sl Transferfunktion} 
$\sigma: \RR \to \RR$ mit einer sogenannten
{\sl Aktivierungsfunktion} $A: \RR^n \to \RR$:
\begin{equation}
\begin{array}{rl}
\kappa: \ \ \ & \RR^n \to \RR^m \ , \\
        & \ \ \ \vec x \mapsto 
        \big( \sigma(A(\vec x)), \sigma(A(\vec x)), \ \ldots \ ,
                               \sigma(A(\vec x)) \big) \ . 
\end{array} 
\end{equation}			       
\end{defi}

\begin{theo}
{\rm (Dichtheitssatz für hyperbolische Aktivierung)} \\
Es sei $f: \RR^n \to \RR$ stetig, $K \subset \RR^n$ kompakt
und $\sigma: \RR \to \RR$ eine stetige Sigmoidalfunktion.
Dann gibt es für alle $\epsilon > 0$ Parameter
\begin{equation}
\begin{array}{rl}
& q \in \NN, \\
& d_{kp} \in \RR, \ \ 1 \le k \le n , \ 1 \le p \le q , \\
& \rho_p \in \RR, \ \ \ 1 \le p \le q , \\
& g_p \in \RR, \ \ \ 1 \le p \le q , 
\end{array}\end{equation}
so dass für alle $\x \in K$ gilt
\begin{equation}
\left| f(\x ) - \sum_{p=1}^q g_p \sigma 
\left( \rho_p \prod\limits_{k=1}^n (x_k-d_{kp}) \right) \right| \ < \ \epsilon .
\end{equation}
Definiert man nun
\begin{equation}
\sigma_{\pi}(\x ) := \sigma(\prod_{k=1}^n x_k) \ ,
\end{equation}
so lässt sich dies auch kurz schreiben als
\begin{equation}
H_{\sigma}  :=	\overline{span} \; \left\{ \sigma_{\pi}(\rho ( \x - \d \, )) \; : \;
\d \in \RR^n, \; \rho \in \RR \right\}	=  C(\RR^n ) \ ,
\end{equation}
wobei $\overline{span}$ den Abschluss bezüglich der Topologie der
gleichmäßigen Konvergenz auf kompakten Mengen bezeichnet.
\end{theo}

\noindent
{\bf Beweis:} Vergleiche die Originalarbeiten \cite{len1,pin} oder
\cite{len3}, S.\ 163ff.
\eop

\noindent
Auch Zeichnungen (vgl.\ Abbildung \ref{bild1}) können sinnvoll sein und
eingebunden werden: 

\begin{figure}[htb]
\centerline{\includegraphics[width=8cm]{/home/lenze/tex/feu/feubild1.ps}}
\caption{\label{bild1}Reales Neuron (Schematische Skizze)}
\end{figure}

\section{Die Topologie neuronaler Strukturen}

\section{Der Ausführ-Modus}

\section{Der Lern-Modus}

\chapter{Implementierung neuronaler Netze}
\label{chap3}
\markboth{\ \ \ \  {\tiny \cpr} \hfill Kap.\ \ref{chap3} \ \ Implementierung neuronaler Netze}{Kap.\ \ref{chap3} \ \ Implementierung neuronaler Netze \hfill {\tiny \cpr} \ \ \ \ }
\section{Die Wahl der Programmiersprache: Java}
Kurze Begründung für die Auswahl der Programmiersprache. Mit welchem
Entwicklungstool wurde gearbeitet und warum? Oder wurde direkt mit dem JDK 
gearbeitet? Wenn ja, warum? {\bf Keine} detaillierte
Einführung in Java; das ist inzwischen Standard. Allerdings: Neue und
spezielle Bibliotheken, Packages oder Klassen, die benutzt werden, müssen
begründet und erläutert werden.

\section{Details zum Entwurf und zur Entwicklung}
Klassische Vorgehensweise bei dem Entwurf und der Entwicklung eines
Anwendungsprogramms (OOA, OOD, OOP, etc.). 
Insbesondere sollten hier (oder -- falls zu umfangreich -- spätestens im
Anhang) die entsprechenden Diagramme eingebunden werden.

\section{Details zur konkreten Implementierung}
Hier sollten ausgewählte Teile des Source-Codes, die für die Funktionalität
des Programms fundamental sind, im Detail erläutert werden. Neben dem
Aufzeigen der generellen Konzepte zur Umsetzung des mathematischen Kalküls in
Programm-Code geht es hier auch um Fragen wie Effizienz, Parallelisierbarkeit,
numerische Stabilität, etc.. 

\section{Probleme bei der Implementierung}
Klar!

\chapter{Anwendung neuronaler Netze}
\label{chap4}
\markboth{\ \ \ \  {\tiny \cpr} \hfill Kap.\ \ref{chap4} \ \ Anwendung neuronaler Netze}{Kap.\ \ref{chap4} \ \ Anwendung neuronaler Netze \hfill {\tiny \cpr} \ \ \ \ }
\section{Anforderungen an Hard- und Software}
\begin{itemize}
\itemsep -6pt
\item Hardware-Plattform
\item Besonderheiten der Rechner-Architektur
\item Prozessor-Typ bzw.\ -Typen 
\item Taktfrequenz 
\item Hauptspeicher-Größe 
\item Cache 
\item Swapspace (falls verwendet)
\item Betriebssystem und Versionsnummer 
\item Entwicklungsumgebung 
\item Programmiersprache 
\item Compiler bzw.\ Interpreter und Versionsnummer 
\item Compiler- bzw.\ Interpreteroptionen 
\item verwendete Bibliotheken
\item alle relevanten Implementationsdetails 
      (insbesondere gesetzte Parameter u.~ä.)
\item usw., usw.
\end{itemize}
Alle diese Angaben sind wichtig, da sich die Ergebnisse sonst nur bedingt
reproduzieren lassen. Nicht reproduzierbare Anwendungen sind aber wertlos!

\section{Anwendung des Java-Programms}

Hier sollte eine Beispiel-Anwendung mit dem entwickelten Programm
durchgespielt werden und möglichst mit Screenshots dokumentiert werden.

\chapter{Zusammenfassung und Ausblick}
\label{chap5}
\markboth{\ \ \ \  {\tiny \cpr} \hfill Kap.\ \ref{chap5} \ \ Zusammenfassung und Ausblick}{Kap.\ \ref{chap5} \ \ Zusammenfassung und Ausblick \hfill {\tiny \cpr} \ \ \ \ }
Noch einmal wird kurz erläutert, was in der Arbeit eigentlich gemacht 
wurde. Außerdem wird gesagt, was nach Meinung der Autorin bzw.\ des
Autors noch alles gemacht werden könnte. Dadurch kann man zeigen, dass
man auch etwas über den Tellerrand der eigentlichen Problemstellung
hinweg gesehen hat. Insgesamt sollten dazu 1--2~Seiten genügen. 
Am Ende dieses Kapitels sollte die Bachelor-/Master-Arbeit etwa 
{\bf 40/80 Seiten} umfassen! \setcounter{page}{40}
Natürlich ist auch dies nur ein grober, aber im Auge zu behaltender
Anhaltspunkt! Lieber gute 40 Seiten als redundante und langweilige 60 Seiten!

\newpage
\markboth{\ \ \ \  {\tiny \cpr} \hfill Literaturverzeichnis}
         {Literaturverzeichnis \hfill {\tiny \cpr} \ \ \ \ }
\normalsize % \footnotesize %\small
\baselineskip=14pt
\begin{thebibliography}{llll}
\addcontentsline{toc}{chapter}{\protect\numberline{}Literaturverzeichnis}
\bibitem{buch} P.P.\ Autor, 
      Titel des Buchs, Verlag, Erscheinungsort, Erscheinungsjahr.

\bibitem{artikel} P.P.\ Autor, Titel des Artikels, 
      {\sl Titel der  Zeitschrift},  
      {\bf Band}, Jahr, ersteSeite--letzteSeite.

\bibitem{wwwlink} P.P.\ Autor, 
      Titel der www-Seite, http-Adresse, Datum.

\bibitem{chu1} C.K.\ Chui und X.\ Li,
      Approximation by ridge functions
      and neural networks with one hidden layer,
      {\sl J.\ Appr.\ Theory} {\bf 70}, 1992, 131--141.

\bibitem{chu2} C.K.\ Chui und X.\ Li,
      Realization of neural networks with
      one hidden layer, in: {\sl Multivariate Approximation: From CAGD to
      Wavelets}, K.\ Jetter und F.\ Utreras (Herausgeber), World Scientific,
      Singapore, 1993, 77--89.

\bibitem{dev} R.\ A.\ DeVore, R.\ Howard und C.\ A.\ Micchelli, 
      Optimal nonlinear approximation,
      {\sl Manuscripta Mathematica} {\bf 63}, 1989, 469--478. 

\bibitem{hec} R.\ Hecht-Nielsen,
      Neurocomputing, Addison-Wesley, Reading (Mas\-sa\-chusetts), 1990.

\bibitem{len1} B.\ Lenze,
      Note on a density question for neural networks, 
      {\sl Numerical Funct.\ Analysis and Optimiz.} {\bf 15}, 1994, 909--913.

\bibitem{len2} B.\ Lenze,
      Einführung in die Fourier-Analysis, Logos Verlag, Berlin, 2000, zweite
      Auflage. 

\bibitem{len3} B.\ Lenze,
      Einführung in die Mathematik neuronaler Netze, Logos Verlag,
      Berlin, 2003, zweite Auflage.

\bibitem{les} M.\ Leshno, V.Y.\ Lin, A.\ Pinkus und S.\ Schocken,
      Multilayer feedforward
      networks with a nonpolynomial activation function can approximate any
      function, {\sl Neural Networks} {\bf 6}, 1993, 861--867.

\bibitem{mha1} H.N.\ Mhaskar,
      Approximation properties of a multilayered feedforward artificial
      neural network, 
      {\sl Adv.\ in Comp.\ Math.} {\bf 1}, 1993, 61--80.

\bibitem{mha2} H.N.\ Mhaskar,
      Neural networks for optimal approximation of smooth and analytic functions,
      {\sl Neural Computation} {\bf 8}, 1996, 164--177.

\bibitem{mha3} H.N.\ Mhaskar und C.A.\ Micchelli,
      Approximation by superposition of sigmoidal
      and radial basis functions,
      {\sl Adv.\ in Appl.\ Math.} {\bf 13}, 1992, 350--373.

\bibitem{mha4} H.N.\ Mhaskar und C.A.\ Micchelli,
      Degree of approximation by neural and translation networks with 
      a single hidden layer,
      {\sl Adv.\ in Appl.\ Math.} {\bf 16}, 1995, 151--183.

\bibitem{mue} B.\ Müller und J.\ Reinhardt,
      Neural Networks, Springer-Verlag, Berlin-Heidelberg-New York, 
      corrected second printing, 1991.

\bibitem{pin} A.\ Pinkus, 
      TDI-subspaces of $C(\RR^d)$ and some density problems from neural networks,
      {\sl J.\ Appr.\ Theory} {\bf 85}, 1996, 269--287.

\end{thebibliography}


\chapter*{}
\addcontentsline{toc}{chapter}{\protect\numberline{}Anhang A}
\markboth{\ \ \ \  {\tiny \cpr} \hfill Anhang}
         {Anhang \hfill {\tiny \cpr} \ \ \ \ }
\normalsize % \footnotesize %\small
\noindent
{\bf \LARGE Anhang A} \\[0.4cm]
Ein oder mehrere Anhänge können, müssen aber nicht vorhanden sein.
Als Faustregel gilt: Alles was den Lesefluss stört, kann in einen
Anhang, also insbesondere Programmlistings (länger als eine Seite),
umfangreiches Tabellen-Material, etc.. Die Listings der Programme,
also der {\bf Original-Source-Code}, 
sollten auf jeden Fall -- außer eventuell im Anhang -- auch auf 
CD-ROM  der Bachelor-/Master-Arbeit in einem Einsteckfach beigelegt
werden. Als Zugabe 
kann dort  auch noch direkt ausführbarer Maschinen-Code für verschiedene
Plattformen hinterlegt  werden. Ebenfalls sollte es eine
Selbstinstallationsroutine für mindestens ein gängiges Betriebssystem auf dem
Datenträger geben! \\ 
Im Gegensatz zu normalen Kapiteln werden Anhänge zur besseren Unterscheidung
nicht mit "`1"', "`2"', "`3"',$\ldots$ durchnummeriert, sondern mit
"`A"', "`B"', "`C"',$\ldots$. Ist nur ein Anhang vorhanden, kann die
Nummerierung "`A"' entfallen. \\[0.2cm]
Am Ende des Anhangs sollte der Umfang der Bachelor-/Master-Arbeit etwa 
{\bf 60/100 Seiten} betragen! 

\chapter*{}
\addcontentsline{toc}{chapter}{\protect\numberline{}Anhang B}
\markboth{\ \ \ \  {\tiny \cpr} \hfill Anhang}
         {Anhang \hfill {\tiny \cpr} \ \ \ \ }
\normalsize % \footnotesize %\small
\renewcommand{\baselinestretch}{1.3} 
\noindent
{\bf \LARGE Anhang B} \\[0.4cm]
\vspace*{2cm}
\begin{center}
Hier könnten Klassen- oder UML-Diagramme stehen!
\end{center}

\chapter*{}
\addcontentsline{toc}{chapter}{\protect\numberline{}Anhang C}
\markboth{\ \ \ \  {\tiny \cpr} \hfill Anhang}
         {Anhang \hfill {\tiny \cpr} \ \ \ \ }
\normalsize % \footnotesize %\small
\renewcommand{\baselinestretch}{1.3} 
\noindent
{\bf \LARGE Anhang C} \\[0.4cm]
\vspace*{2cm}
\begin{center}
Hier könnten konkrete Teile des Java-Quellcodes stehen!
\end{center}

\setcounter{page}{60}

\mbox{}


\chapter*{}
\addcontentsline{toc}{chapter}{\protect\numberline{}Eidesstattliche Erklärung}
\markboth{\ \ \ \  {\tiny \cpr} \hfill Erklärungen}
         {Erklärungen \hfill {\tiny \cpr} \ \ \ \ }
\vspace*{0.5cm}
\noindent
{\bf Eidesstattliche Erklärung} \\
Ich versichere an Eides statt, dass ich die vorliegende Arbeit selbständig
angefertigt und mich keiner fremden Hilfe bedient sowie keine anderen als die
angegebenen Quellen und Hilfsmittel benutzt habe. Alle Stellen, die wörtlich
oder sinngemäß veröffentlichten oder nicht veröffentlichten Schriften und
anderen Quellen entnommen sind, habe ich als solche kenntlich gemacht. Diese
Arbeit hat in gleicher oder ähnlicher Form noch keiner Prüfungsbehörde
vorgelegen. 

\vspace{1cm}
Dortmund, den \today
\ \ \  \ \ \hfill {\tt Unterschrift nicht vergessen!} \\


\vspace*{3cm}
\noindent
{\bf Erklärung} \\
Mir ist bekannt, dass nach § 156 StGB bzw. § 163 StGB eine falsche
Versicherung an Eides Statt bzw. eine fahrlässige falsche Versicherung an
Eides Statt mit Freiheitsstrafe bis zu drei Jahren bzw. bis zu einem Jahr oder
mit Geldstrafe bestraft werden kann. 

\vspace{1cm}
Dortmund, den \today
\ \ \  \ \ \hfill {\tt Unterschrift nicht vergessen!} \\


\newpage
%%%%%%%%%%%%%%%%%%%%%% NICHT IN ARBEIT ÜBERNEHMEN!!! %%%%%%%%%%%%%%%%%%%%%%%%%%
\thispagestyle{empty}
\noindent
\vspace*{6cm}
\begin{center}
{\bf Spezielle Erklärung vor Beginn der Bachelor-Thesis/Master-Thesis}
\end{center}
Hiermit erkläre ich, dass ich die vorausgehenden Seiten, die man sich unter 
\\[0.2cm]
{\scriptsize \bf \hspace*{0.3cm}ftp://gatekeeper.informatik.fh-dortmund.de/pub/professors/lenze/thesis/thesis.pdf} \\[0.2cm]
ansehen kann, mit den Erläuterungen zum Aufbau, zum Umfang und zum Inhalt einer
Bachelor-/Master-Arbeit sorgfältig 
gelesen und verstanden habe. Insbesondere ist mir klar, was man unter
wissenschaftlichem Arbeiten versteht und dass korrektes Zitieren ein
wesentliches Element in diesem Zusammenhang ist. Alle Fragen, die es in diesem
Kontext noch gab, habe ich inzwischen mit Herrn Lenze geklärt, und es bestehen
keine Unklarheiten mehr. Über die besondere Problematik von Plagiaten und den
Kriterien, die ein Vorliegen anzeigen, bin ich ebenfalls genau unterrichtet.

\vspace{1.5cm}
Dortmund, den \\
\ \ \  \ \ \hspace*{8cm} {\tiny Unterschrift!} \\

\vfill

\end{document}





